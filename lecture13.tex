\subsection{Differences in Differences}
In this section, we will make use of panel data, with two periods and two
regions. In the data, the observables are $Y_0,Y_1, R$ (period 0 and period 1
outcome and region), and $X$ (covariates). In our model, the treatment variable
is $D=Rt$. This means that in period 0, no one is treated. In period 1, only
people in region 1 are treated. The model setup is a \emph{natural experiment}.
We can write the model as
\begin{equation*}
    Y_t=Y_t(0)+D\pa{Y_t(1)-Y_t(0)}, \forall t\in\{0,1\}
\end{equation*}
This leads to the following definition for our familiar parameter of interests:
\begin{equation*}
    ATE(x)=ATT(x)=\E\bra{Y_1(1)-Y_1(0)\mid X=x,R=1}
\end{equation*}
Similar to before, we need to impose some sort of independence assumption for identification. Let's call it \textbf{equal trend}
\begin{equation*}\label{eq:equal_trend}
    \E[Y_1(0)-Y_0(0)\mid X,R=1]=\E[Y_1(0)-Y_0(0)\mid X,R=0]
\end{equation*}
which is saying that if there were no treatment, region 1 and region 2 would have the same trend.
Therefore, we can identify the ATE by
\begin{equation} \label{eq:did}
    \begin{split}
        ATE(x)&=\E\bra{Y_1(1)-Y_0(0)+Y_0(0)-Y_1(0)\mid X=x,R=1} \\
        &= \E\bra{Y_1(1)-Y_0(0)\mid X=x,R=1}-\E\bra{Y_1(0)-Y_0(0)\mid X=x,R=0} \\
        &= \E\bra{Y_1-Y_0\mid R=1,X=1}-\E\bra{Y_1-Y_0\mid R=0,X=1}
    \end{split}
\end{equation}

The equation \ref{eq:did} is the \emph{difference in differences} estimator.

As a matter of fact, conditional mean independence \begin{assumption}For all $(t,r,x) \in \set{0,1}\times \pa{\set{0,1}\times supp(X) \cap supp(R,X)}$, we have $\E\bra{Y_t(0) \mid R=r,X=x}=\E\bra{Y_t(0) \mid X=x}$.
\end{assumption}implies equal trend assumption \eqref{eq:did}.

\begin{remark}
    Suppose now we would need a larger vector $\tilde{X}=(X^\top,U^\top)$ to condition on for the conditional mean independence (the treatment is assigned conditioned on observables and unobservables).
    For the above independence to hold and $\E\bra{Y_t(0) \mid X=x, U=u}= \E\bra{Y_t(0) \mid X=x}+c(u)$, \textit{we need that the conditional expectation function is additively separable in $u$}.  Because $c(u)$ is time invariant, the equal trend assumption (condition on $X$ since we can not condition on unobservable $\tilde{X}$) holds while the conditional mean independence assumption (on $X$) doesn't hold.
    This is the usual advantage of panel data compared to only cross-sectional data.
\end{remark}

