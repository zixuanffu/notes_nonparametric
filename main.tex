\documentclass[12pt]{article}
\usepackage{fullpage,graphicx,psfrag,amsmath,amsfonts,verbatim}
\usepackage[small,bf]{caption}
\usepackage{amsthm}
\usepackage{hyperref}
\usepackage{bbm} % for the indicator function to look good
\usepackage{color}
\usepackage{mathtools}
\input newcommand.tex


% \bibliographystyle{alpha}

\title{Econometrics 2: Non-parametric methods }
\author{Eric Gautier\thanks{Notes by FU Zixuan, last compiled on \today.}}
\date{2024 Spring}

\begin{document}
\maketitle

\begin{figure*}[h]
    \centering
    \includegraphics{figures/ihaveaquestion.jpg}
    \caption*{I have a question!}
\end{figure*}



\newpage
\tableofcontents
\newpage

\section{Preliminaries}
\subsection{Probability basics}
\begin{definition}[distribution law]
    The distribution law of a random variable $X$ is $\p_X$ is the probability on $\pa{\Rd,\calb\pa{\Rd}}$ such that  $\p_X\bra{B}=\p\bra{X\in B}$ for all $B\in \calb\pa{\Rd}$
\end{definition}

\begin{definition}[density]
    $X$ has density $f_X$ if $\p_X\bra{B}=\int_B \underbrace{f_X(x)dx}_{d\p_X(x)}$ for all $B\in \calb\pa{\Rd}$
\end{definition}

Let $Y$ be a random variable and $X$ be a random vector in $\Rd$ defined on the same probability space $\pspace$. We want to define and manipulate $\E\bra{Y|X}$.

There are 2 particular cases
\begin{enumerate}
    \item $\pa{Y,X^\top}^\top$ has discrete support. Let $x\in \text{spt}(X)$, then $\E\bra{Y|X=x}=\sum y_j \p\pa{Y=y_j\mid X=X}$. This is well defined only when $\p\pa{X=x}>0$ in which case $\p\pa{Y=y_j\mid X=x}=\frac{\p\pa{Y=y_j,X=x}}{\p\pa{X=x}}$ is well defined.
    \item $\pa{Y,X^\top}^\top$ and $X$ have a density then $\E\bra{Y|X=x}=\int yf_{Y|X=x}(y)dy$ where $f_{Y|X=x}(y)=\frac{f_{Y,X}(y,x)}{f_X(x)}$.
\end{enumerate}
\begin{proposition}[conditional expectation]
    The random variabel $Z \coloneqq \E\bra{Y|X}$ is the unique random variable such that
    \begin{enumerate}
        \item $Z\in L^1\pspace$, that is $Z$ is $\sigma(X)$-measurable.
        \item \label{prop:con_exp_uniqueness}$\E\bra{Z\mathbbm{1}_B}=\E\bra{Y\mathbbm{1}_B}$ for all $B\in \sigma(X)$.
    \end{enumerate}
    \emph{unique} means that if $Z'$ is another random variable satisfying the same properties, then $Z=Z'$ a.s.
\end{proposition}
\begin{remark}
    The random variable $Z$ is $\sigma(X)$-measurable iff $Z=\phi(X)$ for some function $\phi:\pa{\Rd,\calb\pa{\Rd}}\to\pa{\R,\calb\pa{\R}}$. The corresponding function $\phi$ for $Z=\underbrace{E\bra{Y|X}}_{\text{conditional expectation}}$ is denoted by $\underbrace{\E\bra{Y|X=x}}_\text{conditional expectation function}$.
\end{remark}
\begin{remark}
    The proposition \ref{prop:con_exp_uniqueness} is equivalent to 
    \begin{align*}
        &\E\bra{\pa{Y-Z}\mathbbm{1}_B}=0, \quad \forall B\in \sigma(X) \\ \Leftrightarrow &\E\bra{\pa{Y-Z}\psi(X)}=0, \quad \forall \psi\ \text{bounded and meansurable}.
    \end{align*}
\end{remark}

\subsection{Completeness condition}

We want to understand the completeness condition when $X=Z-\eta$, where $Z
    \indep \eta$ and both have densities. Recall the definition of
\textbf{completeness}.
\begin{definition}[Completeness]
    \label{completeness}
    Completeness is defined as such that $$
        \forall z \in \R,\ \int_\R \varphi(x)f_\eta(z-x)dx=0 \text{ implies that for all } x,\ \varphi(x)=0,$$
    where $\varphi$ is continuous and $\int_\R  \abs{{\varphi(x)}}dx <\infty$.
\end{definition}
Now, We make a detour to introduce some notations in function space.
\begin{definition}
    Let $f$ be a function defined on $\Rd$ with values in $\R$ or $\C$ and $p\le 1$. Then $L^p(\Rd)$ is defined as the space of measurable function from $\left(\Rd,\mathcal{B}(\Rd)\right)$ such that $\int_{\Rd}\abs{f(x)}^p dx<\infty$. If $f$ takes value from $\C$, $\abs{\cdot}$ is the modulus.
\end{definition}

\begin{definition} [$L^1\pa{\R}$ space]
    A function is in $L^1\pa{\R}$ if $\int_{\R} \abs{f\pa{x}}dx <\infty$.
\end{definition}
\begin{definition}[Fourier transform]
    If $f\in L^1\pa{\R}$, the Fourier transform of $f$ is defined for all $w\in \R$ by
    \begin{equation*}
        \F\bra{f}\pa{w} = \int_\R e^{iwx} f\pa{x}dx.
    \end{equation*}
\end{definition}
\begin{remark}
    Let $t\in \R$, $e^{it}=\cos(t)+i\sin(t)$ and $\abs{e^{it}}^2=1$
\end{remark}
\begin{definition}[Convolution]
    If $f$ and $g$ belong to $L^1(\Rd)$, the convolution of $f$ and $g$ is $f\ast g(z)=\int f(x)g(z-x)dx$.
\end{definition}
\begin{proposition}
    If $f$ and $g$ belong to $L^1(\Rd)$, then $f\ast g \in L^1(\Rd)$. Its Fourier transformation is $F\bra{f\ast g}(w)=F\bra{f}(w)=F\bra{f}(w)F\bra{g}(w)$ for all $w\in \Rd$
\end{proposition}
\begin{remark}
    check this proposition as an exercise.
\end{remark}
\begin{proposition}
    If $f\in L^1(\Rd)$, then $F[f]$ is continuous and $\lim_{\norm{w}_2 \to \infty} F[f](w)=0$.
\end{proposition}
We introduce two properties that are useful for later cause.
\begin{property}\label{prop:plancherel_inverse}
    If $f,\F\bra{f}\in L^2\pa{\R}\cap L^1\pa{\R}$, then
    \begin{enumerate}
        \item (The Placherel equality)$\frac{1}{2\pi} \norm{\F\bra{f}}^2_2 = \norm{f}^2_2$ (Plancherel's theorem)
        \item (The Fourier inverse formula) For all $x\in \R$, $f\pa{x} = \frac{1}{2\pi} \int_\R e^{-iwx} \F\bra{f}\pa{w}dw$, the inversion of the Fourier transform.
    \end{enumerate}
\end{property}

\begin{question}
    Let $Z\in L^2\pspace$, then $\E [\abs{z}]\le\sqrt{\E [z^2]}\sqrt{\E [1^2]}$. Therefore, $L^2\pspace\subset L^1\pspace$.
\end{question}

\begin{example}
    \label{ex:1}
    Let $K(x)=\frac{1}{\sqrt{2\pi}}e^{-\frac{x^2}{2}}$, then $K\in L^1(\R)\cap L^2(\R)$. Then for all $w\in \R$, $$F[K](w)=e^{-\frac{w^2}{2}}.$$
\end{example}
\begin{example}
    \label{ex:2}
    Let $K(x)=\frac{1}{\sqrt{2}}\mathbbm{1}_{\{ \abs{x}\le 1\}}$, then $K\in L^1(\R)\cap L^2(\R)$. Then for all $w\in \R$,
    \begin{align*}
        F[K](w) & =1/2\int_{-1}^1 \cos(wx)dx+1/2\int_{-1}^1 \sin(wx)dx \\
                & =\frac{1}{2w}[\sin(wx)]\Big\vert_{-1}^1              \\
                & =\frac{\sin(wx)}{w}
    \end{align*}
    Here $F[K]\notin L^1(\R)$ but $F\bra{K}\in L^2(\R)$. Note also that $F[K](w)=0$ if and only if $w=\pm k\pi$ for $k\in \mathrm{N}$.
\end{example}

Let us check whether the functions given in Example~\ref{ex:1} and \ref{ex:2}
satisfy the completeness condition~\ref{completeness} for $X=Z-\eta$.
\begin{enumerate}
    \item Since $F[f_\eta](w)>0$ for all $w$. Thus, $F[\varphi](w)=0\Leftrightarrow
              \varphi(x)=0$ for all $x$.
    \item Similarly, $F[\varphi](w)=0$ for all $w\in \R\setminus S$. Because $\varphi$ is
          continuous, it is $0$ everywhere.
\end{enumerate}
\section{Density function and kernel estimation}
\subsection{Density function}
We want to estimate the density $f_X$ of $X\in \R$ and will work among classes
of densities. For example,
\begin{enumerate}
    \item \textbf{continuous densities}
    \item densities such that for all $x,\ x'\in \R,\ \abs{f_X(x)-f_X(x')}\le
              M\abs{x-x'}$ for some $M>0$
    \item densities which are \textbf{monotonically increasing} on $[0,1]$
\end{enumerate}

\subsection{Density function estimation}
If $X$ has a density $f_X$ , then $f_X(x)=F_X'(x)\ a.e.$ because \begin{equation*}
    F_X(x)=\int_{-\infty}^x f_X(t)dt=\E\bra{\mathbbm{1}_{\{X\le x\}}}.
\end{equation*}
A natural estimator of the CDF is the  \textbf{empirical CDF}, defined as
\begin{equation*}
    \hat{F}_n(x)=\frac{1}{n}\sum_{i=1}^n \mathbbm{1}_{\{X_i\le x\}}.
\end{equation*} where $n$ is the sample size. Therefore, an estimator of $f_X$ is the derivative of the empirical CDF, which is the \textbf{empirical density function} defined as \begin{equation*}
    \begin{split}
        \hat{f}_n(x)=\frac{\hat{F}_X\pa{x+h/2}-\hat{F}_X\pa{x-h/2}}{h}= \frac{1}{nh}\sum_{i=1}^n K\pa{\frac{X_i-x}{h}}
    \end{split}
\end{equation*} where $K(x)=\mathbbm{1}_{\{X\le \frac{1}{2}\}}$
\begin{definition}[kernel function]
    A kernel is a function $K:\R\to \R$ such that $K\in L^1(\R)$ and $\int K(x)dx=1$.
\end{definition}
\begin{definition}[kernel density estimator with kernel $K$ and bandwidth $h$]
    The kernel density estimator of $f_X$ is defined as \begin{equation*}
        \hat{f}_n(x)=\frac{1}{nh}\sum_{i=1}^n K\pa{\frac{X_i-x}{h}}
    \end{equation*}
\end{definition}

\section{Density function and kernel estimation}
\subsection{Density function}
We want to estimate the density $f_X$ of $X\in \R$ and will work among classes of densities. For example,
\begin{enumerate}
    \item \textbf{continuous densities}
    \item densities such that for all $x,\ x'\in \R,\ \abs{f_X(x)-f_X(x')}\le M\abs{x-x'}$ for some $M>0$
    \item densities which are \textbf{monotonically increasing} on $[0,1]$
\end{enumerate}

\subsection{Kernel estimation}
\paragraph{Some kernels} We list out some common kernels.
\begin{enumerate}
    \item \label{ker:rect} $K\pa{x}=\frac{1}{2}\1_{\abs{x}\le \frac{1}{2}}$
  \item\label{ker:gauss} $K\pa{x} = \frac{1}{\sqrt{2\pi}} e^{-\frac{x^2}{2}}$, the Gaussian kernel
  \item\label{ker:sinc} $K\pa{x} = \frac{\sin\pa{x}}{\pi x}$, the sinc kernel
\end{enumerate}
\begin{remark}
    Note that the Gaussian kernel is both in $L^1\pa{\R,\calb\pa{\R},dx}$ and in $L^2\pa{\R,\calb\pa{\R},dx}$. The sinc kernel is only in $L^2\pa{\R,\calb\pa{\R},dx}$ but not in $L^1\pa{\R,\calb\pa{\R},dx}$, as the absolute value fails to be integrable. However, we have
\begin{equation*}
  1 = \lim_{R\rightarrow \infty}\int_{-R}^R \frac{\sin\pa{x}}{\pi x} dx.
\end{equation*}
\end{remark}

\subsection{Performance analysis}
\begin{definition}
  We introduce the quadratic \textbf{risk}
  \begin{equation*}
    \mathrm{MSE}\pa{x} = \E\bra{\pa{\hat{f}_X\pa{x} - f_X\pa{x}}^2},
  \end{equation*}
  where
  \begin{equation*}
    \ell\pa{x,y} = \pa{x-y}^2
  \end{equation*}
  is the \textbf{loss} function.

  Other risks include
  \begin{equation*}
    \E\bra{\sup_{x\in \R} \abs{\hat{f}_X\pa{x}-f_X\pa{x}}} = \E\bra{\norm{\hat{f}_X-f_X }_\infty}
  \end{equation*}
\end{definition}
Note that $\hat{f}_X$ is a function of $x$ and the observations $X= \pa{X_1,\ldots, X_n}$.

\begin{definition}
  We define the \textbf{bias} of $\hat{f}_X\pa{x}$ by
  \begin{equation*}
    \mathrm{Bias}\pa{\hat{f}_X} = b\pa{x} = \E\bra{\hat{f}_X\pa{x} - f_X\pa{x}}
  \end{equation*}
  and we denote the \textbf{variance} of $\hat{f}_X\pa{x}$ by $\sigma^2\pa{x}$.
\end{definition}

\begin{proposition}
  We have
  \begin{equation*}
    \mathrm{MSE}\pa{x} = b\pa{x}^2 + \sigma^2\pa{x}.
  \end{equation*}
\end{proposition}
\begin{proof}
We have
  \begin{equation*}
    \begin{split}
      \MSE\pa{x} &= \E\bra{ \pa{\hat{f}_X\pa{x} {\color{red}- \E\bra{\hat{f}_X\pa{x}} + \E\bra{\hat{f}_X\pa{x}}} - f_X\pa{x}}^2} \\
      &= \E\bra{ \pa{\hat{f}_X\pa{x} - \E\bra{\hat{f}_X\pa{x}}}^2} + 2 \E\bra{\pa{\hat{f}_X\pa{x}- \E\bra{\hat{f}_X\pa{x}}}\underbrace{\pa{\E\bra{\hat{f}_X\pa{x}} - f_X\pa{x}}}_{\text{not random}} } \\
      &\quad + \E\bra{\pa{\E\bra{\hat{f}_X\pa{x}} - f_X\pa{x}}^2} \\
      &= \underbrace{\E\bra{ \pa{\hat{f}_X\pa{x} - \E\bra{\hat{f}_X\pa{x}}}^2}}_{=\sigma^2\pa{x}} + 2 \pa{\E\bra{\hat{f}_X\pa{x}} - f_X\pa{x}} \underbrace{\E\bra{\hat{f}_X\pa{x}- \E\bra{\hat{f}_X\pa{x}}} }_{=0} \\
      &\quad + \underbrace{\pa{\E\bra{\hat{f}_X\pa{x}} - f_X\pa{x}}^2}_{=b\pa{x}^2}. \\
    \end{split}
  \end{equation*}
\end{proof}

\begin{proposition}[upper bound of $\sigma^2(x)$]\label{prop:1}
  Assume that there exists $f_{\max}\in \R$ such that $\forall x\in \R$, $f_X\pa{x}\leq f_{\max}$ and $\int_\R K^2\pa{u}du <\infty$. Then we have, for $C= f_{\max}\int_\R K^2\pa{u}du$,
  \begin{equation*}
    \forall x\in\R\forall n\geq 1 \forall h>0, \sigma^2\pa{x}\leq \frac{C}{nh}.
  \end{equation*}
\end{proposition}
\begin{proof}
  First observe that, by identical distribution of $X_1,\ldots, X_n$,
  \begin{equation*}\label{eq:expectation}
    \E\bra{\hat{f}_X\pa{x}} = \frac{1}{n}\sum_{i=1}^n \frac{1}{h}\E\bra{K\pa{\frac{X_i-x}{h}}} = \frac{1}{h}\E\bra{K\pa{\frac{X_1-x}{h}}}.
  \end{equation*}

  Now, using independence in the second line and identical distribution in the third line,
  \begin{equation*}
  \begin{split}
    \sigma^2\pa{x} &= \E\bra{\pa{\frac{1}{n} \sum_{i=1}^n \pa{\frac{1}{h} K\pa{\frac{X_i-x}{h}}} - \E\bra{\hat{f}_X \pa{x}}}^2} \\
    &=\frac{1}{n^2} \sum_{i=1}^n\E\bra{ \pa{\frac{1}{h} K\pa{\frac{X_i-x}{h}} - \E\bra{\hat{f}_X \pa{x}}}^2}\\
    &= \frac{1}{n} \E\bra{\pa{ \frac{1}{h} K\pa{\frac{X_1-x}{h}}- \E\bra{\hat{f}_X \pa{x}}}^2}
      \end{split}
  \end{equation*}
  Inserting equation* \ref{eq:expectation},
  \begin{equation*}
    \begin{split}
      \sigma^2\pa{x} &= \frac{1}{n} \E\bra{\pa{ \frac{1}{h} K\pa{\frac{X_1-x}{h}} - \E\bra{\frac{1}{h} K\pa{\frac{X_1-x}{h}}}}^2}\\
      &= \frac{1}{n}\Var\bra{\frac{1}{h} K\pa{\frac{X_1-x}{h}}} \\
      &= \frac{1}{n}\pa{ \E\bra{\frac{1}{h^2} K^2\pa{\frac{X_1-x}{h}} } - \E\bra{\frac{1}{h} K\pa{\frac{X_1-x}{h}}}^2} \\
      &\leq \frac{1}{n} \E\bra{\frac{1}{h^2} K^2\pa{\frac{X_1-x}{h}} }\\
      &=\frac{1}{nh} \E\bra{\frac{1}{h} K^2\pa{\frac{X_1-x}{h}} } \\
      &=\frac{1}{nh} \int_\R \frac{1}{h} K^2\pa{\frac{y-x}{h}} f_X\pa{y}dy\\
      &= \frac{1}{nh} \int_\R K^2\pa{u} \underbrace{f_X\pa{x+h u}}_{\leq f_{\max}} du \\
      &\leq \frac{1}{nh} \underbrace{f_{\max} \int_\R K^2\pa{u} du}_{=C},
    \end{split}
  \end{equation*}
  where we used the change of variables $y=x+hu$.
\end{proof}

\begin{definition}[$\beta$ for a density function]
  Let $\beta>0$, $L>0$ and set $\ell = \lfloor \beta\rfloor$, by which we mean the greatest integer \textbf{strictly} less than $\beta$. The Hölder class $\Sigma \pa{\beta,L}$ is the class of functions $f:\R\rightarrow \R$ such that $f^{\pa{\ell}}$ exists and for all $x,x'\in \R$ we have
  \begin{equation*}
    \abs{f^{\pa{\ell}}\pa{x}- f^{\pa{\ell}}\pa{x'}}\leq L\abs{x-x'}^{\beta -\ell}.
  \end{equation*}
\end{definition}

\begin{definition}
  We define
  \begin{equation*}
    \calp\pa{\beta,L} = \set{f\in\Sigma\pa{\beta,L}: f\geq 0, \int_\R f\pa{x}dx =1}.
  \end{equation*}
\end{definition}
\begin{example}
  $\beta =1$ gives the usual Hölder continuity condition: for all $x,x'\in \R$
  \begin{equation*}
    \abs{f\pa{x}-f\pa{x'}}\leq L\abs{x-x'}^{\beta}.
  \end{equation*}
\end{example}
{\color{blue}
\begin{remark}
  This Hölder condition implies continuity of $f$.
\end{remark}}
\begin{definition}[$\beta$ for a kernel]
  $K:\R\rightarrow \R$ is a kernel \textbf{of order $\ell \in \N_0$} if
  \begin{itemize}
    \item $u\mapsto u^j K\pa{u}$ is integrable for any $j\in\set{0,\ldots,\ell}$,
    \item $\int_\R K\pa{u} du =1$,
    \item and $\int_\R u^j K\pa{u} du =0$ for $j\in\set{1,\ldots, \ell}$.
      \end{itemize}
\end{definition}

\begin{proposition}[upper bound of $\abs{b(x)}$]\label{prop:2}
Let $f_X\in \calp\pa{\beta,L}$ with $\beta,L >0$ and $K$ of order $\ell \geq \lfloor \beta \rfloor$ such that
\begin{equation*}
  \int_\R \abs{u}^\beta \abs{K\pa{u}}du <\infty.
\end{equation*}
Then, for all $x\in\R$, $n\geq 1$ and $h>0$, we have
\begin{equation*}
  \abs{b\pa{x}}\leq C_1 h^\beta,
\end{equation*}
where
\begin{equation*}
  C_1 = \frac{L}{\ell !}\int_\R\abs{u}^{\beta} \abs{K\pa{u}} du.
\end{equation*}
\end{proposition}
\begin{proof}
Reusing equation \ref{eq:expectation} and using $1= \int_\R K\pa{u}du$ ,
  \begin{equation*}
    \begin{split}
      b\pa{x} &= \E\bra{\hat{f}_X\pa{x}} - f_X\pa{x} \\
      &= \frac{1}{h}\E\bra{K\pa{\frac{X_1-x}{h}}} - f_X\pa{x} \\
      &= \frac{1}{h}\int_\R K\pa{\frac{y-x}{h}} f_X\pa{y}dy - f_X\pa{x}\\
      &= \frac{1}{h}\int_\R K\pa{\frac{y-x}{h}} f_X\pa{y}dy - \int_\R K\pa{u} f_X\pa{x} du.
    \end{split}
  \end{equation*}
  With the change of variables $y = hu + x$, we obtain
  \begin{equation*}
    \begin{split}
      b\pa{x} &= \int_\R K\pa{u} f_X\pa{hu+x}du - \int_\R K\pa{u} f_X\pa{x} du \\
      &= \int_\R K\pa{u} \pa{f_X\pa{hu +x} - f_X\pa{x}} du.
    \end{split}
  \end{equation*}
  By a Taylor expansion, for some $\tau \in [0,1]$, we obtain
  \begin{equation*}
    f_X\pa{hu+x} - f_X\pa{x} = uh f_X'\pa{x} + \cdots  + \frac{\pa{uh}^{\ell -1}}{\pa{\ell -1}!} f_X^{\pa{\ell -1}}\pa{x}+ \frac{\pa{uh}^\ell}{\ell !}f_X^{\pa{\ell}}\pa{x+\tau uh }.
  \end{equation*}
  Thus, recalling that $\int_\R u^j K\pa{u} du = 0$ for $j\in \set{1,\ldots, \ell}$ (we use it in the second and the third step),
  \begin{equation*}
  \begin{split}
    b\pa{x} &= \int_\R K\pa{u} \pa{uh f_X'\pa{x} + \cdots  + \frac{\pa{uh}^{\ell -1}}{\pa{\ell -1}!} f_X^{\pa{\ell -1}}\pa{x}+ \frac{\pa{uh}^\ell}{\ell !}f_X^{\pa{\ell}}\pa{x+\tau uh }} du\\
    &= \int_\R K\pa{u} \frac{\pa{uh}^\ell}{\ell !}f_X^{\pa{\ell}}\pa{x+\tau uh } du \\
    &= \int_\R K\pa{u} \frac{\pa{uh}^\ell}{\ell !} \pa{ f_X^{\pa{\ell}}\pa{x+\tau uh } - f_X^{\pa{\ell}}\pa{x}} du.
    \end{split}
  \end{equation*}
  Taking absolute values, using the Hölder property $f_X\in \calp\pa{\beta, L}$, and recalling finally $0\leq \tau \leq 1$,
  \begin{equation*}
    \begin{split}
      \abs{b\pa{x}} &= \abs{\int_\R K\pa{u} \frac{\pa{uh}^\ell}{\ell !} \pa{ f_X^{\pa{\ell}}\pa{x+\tau uh } - f_X^{\pa{\ell}}\pa{x}} du} \\
      &\leq \int_\R \abs{K\pa{u}} \frac{\abs{uh}^\ell}{\ell !} \abs{ f_X^{\pa{\ell}}\pa{x+\tau uh } - f_X^{\pa{\ell}}\pa{x}} du \\
      &\leq \int_\R \abs{K\pa{u}} \frac{\abs{uh}^\ell}{\ell !} L\abs{\tau u h}^{\beta - \ell} du \\
      &= \int_\R \abs{K\pa{u}}  \frac{L\abs{uh}^{\beta}}{\ell !} \abs{\tau}^{\beta - \ell} du \\
      &\leq  \int_\R \abs{K\pa{u}}  \frac{L\abs{uh}^{\beta}}{\ell !} du \\
      &= \frac{L h^\beta}{\ell !}\int_\R \abs{K\pa{u}}\abs{u}^{\beta} du.   \end{split}
  \end{equation*}
  This shows the claim.
\end{proof}
\begin{remark}
Note that the expectation
  \begin{equation*}
  \E\bra{\hat{f}_X\pa{x}} = \frac{1}{h}\int_\R K\pa{\frac{y-x}{h}} f_X\pa{y}dy
  \end{equation*}
  is the \textbf{convolution} $\frac{1}{h} K\pa{\frac{- \pa{\cdot}}{h}}\ast f_X$.

  In general, the convolution of two integrable functions $f,g:\R\rightarrow \R$ is defined as
  \begin{equation*}
  \pa{f\ast g}\pa{x} = \int_\R f\pa{x-y} g\pa{y} dy.
    \end{equation*}

  One interpretation of the convolution is the following: if $f_X,f_Y$ are the densities of independent random variables $X,Y$, then the density of $X+Y$ is $f_X\ast f_Y$.

  Indeed, let $\varphi$ be bounded and continuous. Then, using independence and writing $u=x+y$, and using Fubini-Tonelli,
  \begin{equation*}
  \begin{split}
    \E\bra{\varphi\pa{X+Y}}&=\int_{\R\times \R} \varphi \pa{x+y} f_{X,Y}\pa{x,y} dx dy \\
    &= \int_\R \int_\R \varphi\pa{x+y}f_X\pa{x}f_Y\pa{y} dx dy \\
    &= \int_\R \int_\R \varphi\pa{u} f_X\pa{y-u} f_Y\pa{y} du dy \\
    &= \int_\R \int_\R \varphi\pa{u} f_X\pa{y-u}f_Y\pa{y} dy du \\
    &=\int_\R \varphi\pa{u}\int_\R f_X\pa{y-u}f_Y\pa{y} dy du \\
    &=\int_\R \varphi\pa{u} f_X\ast f_Y \pa{u} du.
    \end{split}
  \end{equation*}
  This characterises the density uniquely.

  Another way to see this is to consider the characteristic function, which is the Fourier transform of the random variable, using independence:
  \begin{equation*}
    \E\bra{e^{it\pa{X+Y}}} = \E\bra{e^{itX}e^{itY}} = \E\bra{e^{itX}}\E\bra{e^{itY}}.
  \end{equation*}
  The latter is the product of the characteristic functions of $X$ and $Y$. The very same expression as on the right-hand side is yielded taking the characteristic function of a random variable with density $f_X\ast f_Y$, and the characteristic function characterises the distribution uniquely.
\end{remark}
\paragraph{Result}
  Combining proposition \ref{prop:1} and \ref{prop:2}, we see
  \begin{equation*}
    \MSE \pa{x}\leq C_1^2 h^{2\beta} + \frac{C}{nh}.
  \end{equation*}
  Minimizing the right-hand side in $h$ yields $h_{\opt}= \pa{\frac{C}{2\beta C_1^2 n}}^{\frac{1}{2\beta +1}}\sim n^{-\frac{1}{2\beta +1}}$.

  Plugging this back into the right-hand side, we obtain
  \begin{equation*}
    \MSE\pa{x} = O\pa{n^{-\frac{2\beta}{2\beta+1}}}.
  \end{equation*}


% \begin{proposition}
%     Let $f_X \in P(\beta,L) with \beta,L\ge 0 and K of order l\ge [\beta] such that \int\abs{K(\mu)} < \infty. Then \forall $
% \end{proposition}

\section{MISE and Cross validation}
\subsection{MISE}
To define the $\MISE$, we would like
\begin{equation*}
	\E\bra{\norm{\hat{f}_X - f_X}^2_2} < \infty.
\end{equation*}
We assume $f_X\in L^2\pa{\R}$. We would like as well $\hat{f}_X\in L^2\pa{\R}$. This is true if $K\in L^2\pa{\R}$.

Indeed,
\begin{equation*}
	\begin{split}
		\norm{\hat{f}_X}_2^2 & \leq \frac{2^{n-1}}{\pa{nh}^2} \sum_{i=1}^n \int_\R K\pa{\frac{X_i -x}{h}}^2 dx \\
		                     & \leq \frac{2^{n-1}}{nh} \int_{\R} K^2\pa{u} du <\infty.
	\end{split}
\end{equation*}
The idea behind this inequality is $\pa{a+b}^2 \leq 2 \pa{a^2 + b^2}$, and then by induction, $\pa{\sum_{i=1}^n a_i }^2 \leq 2^{n-1} \sum_{i=1}^n a_i^2$.
\subsection{Cross validation}
Let us write
\begin{equation*}
	\begin{split}
		\mathrm{MISE}\pa{h} & = \E\bra{\int_\R \pa{\hat{f}^h_X\pa{x}-f_X\pa{x} }^2dx}                                                                                  \\
		                    & = \underbrace{\E\bra{\int_\R \pa{\hat{f}^h_X \pa{x}}^2 dx - 2\int_\R \hat{f}^h_X\pa{x} f_X\pa{x}dx }}_{=I\pa{h}} + \int_\R f_X^2\pa{x}dx
	\end{split}
\end{equation*}
% \cite{rudemo} 
introduced
\begin{equation*}
	\widehat{\mathrm{CV}}\pa{h} = \int_\R \hat{f}_X^2 \pa{x}dx - \underbrace{\frac{2}{n} \sum_{i=1}^n \hat{f}_{X,-i}\pa{X_i}}_{=\hat{A}},
\end{equation*}
where $\hat{f}_{X,-i}\pa{x} = \frac{1}{\pa{n-1}h} \sum_{j=1,j\neq i}^n K\pa{\frac{X_j-x}{h}}$. The cross-validated bandwidth is
\begin{equation*}
	\hat{h}_{\mathrm{CV}} = \argmin_{h>0 }\widehat{\mathrm{CV}}\pa{h}
\end{equation*}
We claim
\begin{equation*}
	\frac{1}{2}\E\bra{\hat{A}}= \E\bra{\int_\R \hat{f}_X\pa{x} f_X\pa{x} dx}
\end{equation*}
We have
\begin{equation*}
	\begin{split}
		\E\bra{\hat{f}_{X,-1}\pa{X_1}} & = \E_{\p_{X_2}\otimes  \cdot\otimes \p_{X_n}}\bra{\int_\R \hat{f}_{X,-i}\pa{x}f_X\pa{x} dx} \\
		                               & = \E\bra{\frac{1}{\pa{n-1} h}\sum_{j=2}^n \int_\R K\pa{\frac{X_j-x}{h}}f_X\pa{x} dx}        \\
		                               & =\frac{1}{h} \int_\R\int_\R K\pa{\frac{z-x}{h}} f_X\pa{z} f_X\pa{x} dz dx
	\end{split}
\end{equation*}
As an exercise, show that this yields the claim.

\begin{theorem}[Oracle inequality]
	\label{thm:dalelane}
	Let $f_{\max}$ be such that for all $x$, $f_X\pa{x}\leq f_{\max} <\infty$. Assume the kernel $K$ is such that
	$\int_\R K^2\pa{u}du<\infty$. $\F\bra{K}\geq 0$ and $\mathrm{supp}\pa{\F\bra{K}}\subseteq [-1,1]$. Then $\hat{f}_X^\ast = \hat{f}_X^{h_{\mathrm{CV}}}$ is such that for all $0<\delta <1$, for all $n\geq 1$,
	\begin{equation*}
		\E\bra{\int_\R \pa{\hat{f}^\ast_X \pa{x} - f_X\pa{x}}^2dx} \leq \pa{1+\frac{C}{n^\delta}} \min_{h>\frac{1}{n}} \E\bra{\int \pa{\hat{f}_X^h\pa{x} - f_X\pa{x}}^2 dx} + \frac{C\pa{\log n}^{\frac{\delta}{2}}}{n^{1-\delta}}
	\end{equation*}
\end{theorem}
\begin{remark}
	The cross-validation bandwidth from theorem \ref{thm:dalelane} is random. The kind of inequality in the the theorem is called \textbf{oracle inequality}, as it is not possible to obtain the values on each side. They involve the unknown $f_X\pa{x}$. The estimation of errors in cross-validation kernel estimation is hard, but in practice it often works well.
\end{remark}

\section{Sobolev class and symmetric kernel}
\subsection{Review of Fourier transform}
\begin{definition}
	The characteristic function of a random variable $X$ is
	\begin{equation*}
		\varphi_X\pa{w} = \E\bra{e^{iwX}} = \int_\R e^{iwx} f_X\pa{x}dx.
	\end{equation*}
\end{definition}

\begin{remark}\label{rem:fourier_l2}
	It is possible as well to define the Fourier transform of $f\in L^2\pa{\R}$. Therefore, we take a sequence $f_m \in L^1\pa{\R}\cap L^2\pa{\R}$ such that $\norm{f-f_m}^2_2\rightarrow 0$ as $m\rightarrow \infty$ and define the Fourier transform of $f$ as the $L^2$-limit of $\F \bra{f_m}$. More precisely, we may take $f_m\pa{x} = f\pa{x} \one_{\abs{x}\leq m}$. It is in $L^2$ as the product of an $L^2\pa{\R}$ function and a bounded function, and it is in $L^1\pa{\R}$ as a result of the Cauchy-Schwarz inequality:
	\begin{equation*}
		\int_{\R} f\pa{x} \one_{\abs{x}\leq m} dx \leq \sqrt{\int_{\R} f\pa{x}^2 dx } \sqrt{\int_{-m}^m 1 dx } = \sqrt{2m}  \sqrt{\underbrace{\int_{\R} f\pa{x}^2 dx}_{<\infty} }.
	\end{equation*}
	Moreover,
	\begin{equation}\label{eq:cauchy}
		\norm{f_m -f}^2_2 = \int_{-\infty}^m \abs{f\pa{x}}^2 dx + \int_m^\infty \abs{f\pa{x}}^2 dx \rightarrow 0
	\end{equation}
	as $m\rightarrow \infty$.
	% \begin{exercise}
	% 	If $f$ is symmetric, $\F \bra{f}$ is real-valued.
	% \end{exercise}
	By equation \ref{eq:cauchy}, for all $m,m', \norm{f_m -f_{m'}}_2^2\rightarrow
		0$ as $m,m'\rightarrow \infty$, i.e.~$\pa{f_m}$ is a Cauchy sequence. By
	Plancherel's theorem \ref{prop:plancherel_inverse},
	\begin{equation*}
		\norm{\F \bra{f_m} -\F\bra{f_{m'}} }_2^2 = \norm{\F \bra{f_m -f_{m'} }}_2^2 = 2\pi \norm{f_m -f_{m'}}_2^2.
	\end{equation*}
	Thus, $\F\bra{f_m}$ is a Cauchy sequence in $L^2\pa{\R}$, so that it admits a limit in $L^2\pa{\R}$, since $L^2\pa{\R}$ is a complete normed space. We can then define the Fourier transform of $f$ to be this limit.
\end{remark}
% \begin{exercise}
% 	Prove
% 	\begin{enumerate}
% 		\item $\F\bra{f\pa{\cdot}} \pa{w} = a \F \bra{f\pa{\cdot}}\pa{w}$,
% 		\item $\F\bra{\frac{1}{h}f\pa{\frac{\pa{\cdot}}{h}}}\pa{w} = \F\bra{f\pa{\cdot}}\pa{hw}$,
% 		\item $\F\bra{f\pa{t-\cdot}}\pa{w} = e^{iwt} \F\bra{f\pa{\cdot}} \pa{-w}$.
% 	\end{enumerate}
% \end{exercise}
% \begin{exercise}
% 	If $\hat{f}_X$ is a kernel density estimator, where the kernel is symmetric, then
% 	\begin{enumerate}
% 		\item $\F\bra{K}$ is symmetric and real,
% 		\item $\F\bra{\hat{f}_X}\pa{w} = \phi_u \pa{w} \F\bra{K} \pa{hw}$,
% 	\end{enumerate}
% 	where $\phi_u \pa{w} = \frac{1}{n}\sum_{j=1}^n e^{iwX_j}$.
% \end{exercise}
% \begin{exercise}
% 	By 2 of \ref{prop:plancherel_inverse}, check that the Fourier transform of the sinc kernel is $\one_{\abs{w}\leq 1}$.
% \end{exercise}

% Let $f\in L^1\pa{\R}$ such that $\F\bra{f}\in L^1\pa{\R}\cap L^2\pa{\R}$. Then
% \begin{equation*}
% 	f\pa{x} = \frac{1}{2\pi} \int_{\R} e^{-iwx}\F\bra{f}\pa{w} d w.
% \end{equation*}

% We can define, after differentiating in the usual way without knowing whether the right-hand side is differentiable,
% \begin{equation*}
% 	f'\pa{x} \defeq \frac{i}{2\pi} \int_{\R} \pa{-iw} e^{-iwx} \F\bra{f}\pa{w}dw,
% \end{equation*}
% and
% \begin{equation*}
% 	f''\pa{x} \defeq \frac{1}{2\pi} \int_\R \pa{w^2} e^{-iwx} \F\bra{f} \pa{w}dw,
% \end{equation*}
% and so on. 
We notice that the characteristic function of a random variable is the Fourier
transform of its density. Therefore, the density function $f_X$ of a random
variable $X$ is the inverse Fourier transform of its characteristic function
$\varphi_X$.
\begin{equation*}
	f_X\pa{x} = \frac{1}{2\pi}^d \int_{\R^d} e^{-iwx} \varphi_X\pa{w}dw.
\end{equation*}
We can get the derivative of the density function by differentiating the Fourier transform of the density function.
\begin{equation*}
	f_X^{(m)}\pa{x}= \frac{1}{2\pi^d} \int_{\R^d} \pa{-iw}^m e^{-iwx} \varphi_X\pa{w}dw.
\end{equation*}
It means that $\F\bra{f_X^{(m)}}\pa{w} = \pa{-iw}^m \varphi_X\pa{w}.$

\subsection{Sobolev class}
Building on this, we make the following definition.
\begin{definition}[Sobolev class]\label{def:sobolev_class}
	Let $\beta>0$, $L>0$, the Sobolev class $\calp_S\pa{\beta,L}$ is defined as
	\begin{equation*}
		\calp_S\pa{\beta,L} = \set{f: f \text{ is a density on }  \R \text{ and } \int_{\R}\abs{w}^{2\beta} \abs{\F\bra{f}\pa{w}}^2 dw \leq 2\pi L^2}.
	\end{equation*}
\end{definition}
The restriction is basically saying that the $L^2$ norm of the function $f_X^{(m)}$ is bounded by $L$. This is a generalization of the $L^2$ norm of the function $f_X\pa{x}$, which is the $L^2$ norm of the density function $f$.

%%%
\subsection{Symmetric kernel}
\begin{theorem}[Symmetric kernel]\label{thm:sym_ker}
	Let $f_X\in L^2\pa{\R}, K\in L^2\pa{\R}$ be a symmetric kernel such that
	\begin{equation*}\label{eq:tocheck}
		\sup_{w\in \R\setminus \set{0}} \frac{\abs{1- \F \bra{K} \pa{w}}}{\abs{w}^{\beta'}}\leq A <\infty
	\end{equation*}
	for some $\beta',A>0$. Then
	\begin{equation*}
		\sup_{f_X\in \calp_S \pa{\beta, L}} \E\bra{\norm{\hat{f}_X - f_X}^2_2}\leq C n^{-\frac{2\tilde{\beta}}{2\tilde{\beta}+1}},
	\end{equation*}
	where $\tilde{\beta}=\min \set{\beta,\beta'}$, if $h= \alpha n^{-\frac{1}{2\tilde{\beta} +1}}$ for some $\alpha >0$ and $C$ is a constant which only depends on $L,\alpha,A,K$.
\end{theorem}

\begin{example}\label{exa:fourier_kernel}
	\begin{enumerate}
		\item Gaussian kernel: $K\pa{u} = \frac{1}{\sqrt{2\pi}} e^{-\frac{u^2}{2}}$,
		      $\F\bra{K}\pa{u} = e^{-\frac{u^2}{2}}$. We have
		      \begin{equation*}
			      \frac{\abs{1-e^{-w^2 /2 }}}{\abs{w}^{\beta'}} \leq \begin{cases} \abs{w}^{-\beta'},              & \abs{w}\geq 1 \\
              \frac{w^2/2}{\abs{w}^{\beta'}}, & \abs{w}\leq 1\end{cases}
		      \end{equation*}
		      so \ref{eq:tocheck} holds if $\beta'\leq 2$, else the $\sup$ is $\infty$.
		\item The sinc kernel: $K\pa{u} = \frac{\sin\pa{u}}{\pi u}, \F\bra{K}\pa{w}=
			      \one_{\abs{w}\leq 1}$. We have
		      \begin{equation*}
			      \frac{\abs{1- \F\bra{K}\pa{w}}}{\abs{w}^{\beta'}} \leq \begin{cases}
				      \abs{w}^{-\beta'}, & \abs{u}> 1     \\
				      0,                 & \abs{u}\leq 1,
			      \end{cases}
		      \end{equation*}
		      so \ref{eq:tocheck} holds for all $\beta'$. Such a kernel is called an \textbf{infinite power kernel} or \textbf{superkernel}.
		\item Trapeze kernel: Let \begin{equation*}
			      \F\bra{K}\pa{w} = \begin{cases}
				      0,             & \abs{w} >a        \\
				      1,             & \abs{w}\leq b     \\
				      \text{linear}, & \text{otherwise},
			      \end{cases}
		      \end{equation*}
		      a trapeze. Then \ref{eq:tocheck} holds for all $\beta'$.
		      Let us write $K_2$ for the trapeze (in Fourier space) and $K_1$ for the sinc Kernel (see \ref{exa:fourier_kernel}). Then
		      \begin{equation*}
			      K_2 = \frac{1}{2\pi} \F\bra{\F\bra{K_1}\ast F\bra{K_1}} = \frac{1}{2\pi} \F\bra{\F\bra{K_1}}\F\bra{\F\bra{K_1}} = 2\pi K_1^2\pa{u} = 2\pi \pa{\frac{\sin u}{\pi u}},
		      \end{equation*}
		      which is in $L^1\pa{\R}\cap L^2\pa{\R}$.
	\end{enumerate}
\end{example}
\paragraph{Optimal rate of convergence} It can be shown that the \emph{sinc} kernel has the optimal rate of
convergence.

A Corollary of the Theorem \ref{thm:sym_ker} that we have seen for
cross-validation is
\begin{corollary}
	Let $K$ be the sinc kernel, then
	\begin{equation*}
		\sup_{f_X\in \calp_S\pa{\beta,L}}\E\bra{\norm{\hat{f}_X^{\CV} - f_X}_2^2 }\leq Cn^{-\frac{2\beta}{2\beta +1}}
	\end{equation*}
	for all $\beta > \frac{1}{2}, L>0$, where $C$ only depends on $\beta$ and $L$.
\end{corollary}
Some people have shown:
\begin{proposition}
	\begin{equation*}
		\inf_{\hat{f}}\sup_{f_X\in \calp_S\pa{\beta,L}}\E\bra{\norm{\hat{f}_X - f_X}^2_2}\geq C_\ast n^{-\frac{2\beta}{2\beta +1}}
	\end{equation*}
	for some absolute constant $C_\ast$.
\end{proposition}

This means that $n^{-\frac{2\beta}{2\beta +1 }}$ is the ``minimax'' optimal
rate of convergence and the cross-validated estimator is minimax adaptive
(i.e.~we can construct it with the data only).

\paragraph{Kernel comparison} We end this section by the following table.
\begin{table}[!h]
	\centering
	\begin{tabular}{l|c|c|c}
		name         & kernel & $\F\bra{K}$ & \ $\frac{\abs{1- \F \bra{K} \pa{w}}}{\abs{w}^{\beta}}$ \\
		\hline
		Gaussian     &        &             &                                                        \\
		Epanechnikov &        &             &                                                        \\
		Sinc         &        &             &                                                        \\
		Trapeze      &        &             &                                                        \\
	\end{tabular}
	\caption{Summary}
\end{table}

%%%
\section{MISE and Cross validation}
\subsection{MISE}
To define the $\MISE$, we would like
\begin{equation*}
	\E\bra{\norm{\hat{f}_X - f_X}^2_2} < \infty.
\end{equation*}
We assume $f_X\in L^2\pa{\R}$. We would like as well $\hat{f}_X\in L^2\pa{\R}$. This is true if $K\in L^2\pa{\R}$.

Indeed,
\begin{equation*}
	\begin{split}
		\norm{\hat{f}_X}_2^2 & \leq \frac{2^{n-1}}{\pa{nh}^2} \sum_{i=1}^n \int_\R K\pa{\frac{X_i -x}{h}}^2 dx \\
		                     & \leq \frac{2^{n-1}}{nh} \int_{\R} K^2\pa{u} du <\infty.
	\end{split}
\end{equation*}
The idea behind this inequality is $\pa{a+b}^2 \leq 2 \pa{a^2 + b^2}$, and then by induction, $\pa{\sum_{i=1}^n a_i }^2 \leq 2^{n-1} \sum_{i=1}^n a_i^2$.
\subsection{Cross validation}
Let us write
\begin{equation*}
	\begin{split}
		\mathrm{MISE}\pa{h} & = \E\bra{\int_\R \pa{\hat{f}^h_X\pa{x}-f_X\pa{x} }^2dx}                                                                                  \\
		                    & = \underbrace{\E\bra{\int_\R \pa{\hat{f}^h_X \pa{x}}^2 dx - 2\int_\R \hat{f}^h_X\pa{x} f_X\pa{x}dx }}_{=I\pa{h}} + \int_\R f_X^2\pa{x}dx
	\end{split}
\end{equation*}
% \cite{rudemo} 
introduced
\begin{equation*}
	\widehat{\mathrm{CV}}\pa{h} = \int_\R \hat{f}_X^2 \pa{x}dx - \underbrace{\frac{2}{n} \sum_{i=1}^n \hat{f}_{X,-i}\pa{X_i}}_{=\hat{A}},
\end{equation*}
where $\hat{f}_{X,-i} = \frac{1}{\pa{n-1}h} \sum_{j=1,j\neq i}^n K\pa{\frac{X_j-x}{h}}$. The cross-validated bandwidth is
\begin{equation*}
	\hat{h}_{\mathrm{CV}} = \argmin_{h>0 }\widehat{\mathrm{CV}}\pa{h}
\end{equation*}
We claim
\begin{equation*}
	\frac{1}{2}\E\bra{\hat{A}}= \E\bra{\int_\R \hat{f}_X\pa{x} f_X\pa{x} dx}
\end{equation*}
We have
\begin{equation*}
	\begin{split}
		\E\bra{\hat{f}_{X,-1}\pa{X_1}} & = \E_{\p_{X_2}\otimes  \cdot\otimes \p_{X_n}}\bra{\int_\R \hat{f}_{X,-i}\pa{x}f_X\pa{x} dx} \\
		                               & = \E\bra{\frac{1}{\pa{n-1} h}\sum_{j=2}^n \int_\R K\pa{\frac{X_j-x}{h}}f_X\pa{x} dx}        \\
		                               & =\frac{1}{h} \int_\R\int_\R K\pa{\frac{z-x}{h}} f_X\pa{z} f_X\pa{x} dz dx
	\end{split}
\end{equation*}
As an exercise, show that this yields the claim.

\begin{theorem}[Oracle inequality]
	\label{thm:dalelane}
	Let $f_{\max}$ be such that for all $x$, $f_X\pa{x}\leq f_{\max} <\infty$. Assume the kernel $K$ is such that
	$\int_\R K^2\pa{u}du<\infty$. $\F\bra{K}\geq 0$ and $\mathrm{supp}\pa{\F\bra{K}}\subseteq [-1,1]$. Then $\hat{f}_X^\ast = \hat{f}_X^{h_{\mathrm{CV}}}$ is such that for all $0<\delta <1$, for all $n\geq 1$,
	\begin{equation*}
		\E\bra{\int_\R \pa{\hat{f}^\ast_X \pa{x} - f_X\pa{x}}^2dx} \leq \pa{1+\frac{C}{n^\delta}} \min_{h>\frac{1}{n}} \E\bra{\int \pa{\hat{f}_X^h\pa{x} - f_X\pa{x}}^2 dx} + \frac{C\pa{\log n}^{\frac{\delta}{2}}}{n^{1-\delta}}
	\end{equation*}
\end{theorem}
\begin{remark}
	The cross-validation bandwidth from theorem \ref{thm:dalelane} is random. The kind of inequality in the the theorem is called \textbf{oracle inequality}, as it is not possible to obtain the values on each side. They involve the unknown $f_X\pa{x}$. The estimation of errors in cross-validation kernel estimation is hard, but in practice it often works well.
\end{remark}


\subsection{Extension}
\begin{remark}
    The condition \ref{thm:sym_ker} is satisfied for an integer $\beta$ if $K$ is a kernel of order $\beta-1$ and $\int \abs{u}^\beta \abs{K\pa{u}}<\infty$.
\end{remark}
\begin{remark}
    We can work with a smaller class of \textit{super smooth} density functions.
    \begin{enumerate}
        \item $\calp_{\alpha,r}=\set{f\in L^2(\R) \quad \text{such that} \quad \int \exp(\alpha\abs{w}^2)\abs{\phi(w)}^2 dw \le L^2}$ where $\phi=\F\bra{f}$ is the Fourier transform of $f$. We can show that a MISE optimal kernel density estimatro could have a risk less than $C\frac{(\log n)^{1/r}}{n}$.
        \item $\calp_{\alpha,r}=\set{f\in L^2(\R) \quad \text{such that} \quad \text{supp}\pa{\phi}\subset\bra{-a,a}}$. In this case, the upper bound is $\frac{a\pi}{n}$.
    \end{enumerate}
\end{remark}

\section{Other types of non-parametric estimators}
\subsection{Orthogonal series estimators}\footnote{Generalizations are called sieves (in Econometrics) or dictionaries in machine-learning.}
Let $f_X\in L^2\pa{\bra{0,1}^d}$, where $L^2\pa{\bra{0,1}^d}$ can be proven to be a \textit{separable Hilbert space} when endowed with the inner product
\begin{equation*}
\angs{f,g}=\int_\R f(x)g(x) dx.
\end{equation*}
We write
\begin{equation*}
\norm{f}_2^2 = \angs{f,f}.
\end{equation*}
{Some properties are comparable to $\R^d$ with $\angs{x,y} = x^T y$.} As a separable space, $L^2\pa{[0,1]^d}$ has a countable basis $\pa{e_j}_{j=1}^\infty$, which is a sequence of functions in $L^2\pa{[0,1]^d}$ such that for all
\begin{equation*}
\angs{e_j,e_k} =\delta_{jk} = \begin{cases}
1, &j=k,\\
0, &\text{else},
\end{cases}
\end{equation*}
and for all $f\in L^2\pa{\bra{0,1}^d}$,
\begin{equation*}
f=\lim_{k\rightarrow\infty} \sum_{j=1}^k \angs{f,e_j} e_j.
\end{equation*}
{Think of $\R^d$, where $\pa{e_j}_{j=1}^d$ is a basis for $e_j =\pa{0,\ldots,0,1,0,\ldots,0}$ the $j$-th unit vector. Then $\angs{e_j,e_k} = e_j^T e_k = \delta_{jk}$, and for $x\in \R^d$,
\begin{equation*}
  x = \sum_{j=1}^d x_j e_j = \sum_{j=1}^d x^T e_j e_j = \sum_{j=1}^d \angs{x,e_j} e_j.
\end{equation*}}
Given $\pa{e_j}_{j=1}^\infty$ a basis, for all $f\in L^2\pa{[0,1]^d}$, \begin{equation*}
\norm{f}^2_2=\sum_{j=1}^\infty \angs{f,e_j}^2.
\end{equation*}
This is a version of the Pythagorean theorem. {In $\mathbb{R}^d$,
\begin{equation*}
\norm{x}^2_2 = \sum_{j=1}^d x_j^2 = \sum_{j=1}^d \angs{x,e_j}^2.
\end{equation*}}
Back to our goal to estimate $f_X = \lim_{k\rightarrow \infty}\sum_{j=1}^\infty \angs{f,e_j} e_j$. For some $T\in \N$, consider $f_X^T \defeq \sum_{j=1}^T \angs{f,e_j} e_j$. The idea is to estimate this cut-off sum instead of the limit expression for $f_X$. We have
\begin{equation*}
c_j \defeq \angs{f_X,e_j}=\int_{[0,1]^d}f_X(x)e_j(x)dx = \E\bra{e_j(X)},
\end{equation*}
so that an unbiased estimator is
\begin{equation*}
\hat{c}_j = \frac{1}{n}\sum_{i=1}^n e_j\pa{X_i}.
\end{equation*}
Thus, a candidate estimator for $f_X$ is
\begin{equation*}
  \hat{f}_X^T = \sum_{j=1}^T \hat{c}_j e_j,
\end{equation*} where
\begin{equation*}
\E\bra{\hat{f}_X^T}=\sum_{j=1}^T c_j e_j = f_X^T.
\end{equation*}
It is possible to write
\begin{equation*}
  \hat{f}_X^T=\frac{1}{n} \sum_{i=1}^n \underbrace{\sum_{j=1}^T e_j(X_i)e_j(x)}_{q_T\pa{X_i,x}},
\end{equation*}
where $q_T\pa{X_i,x}$ plays the role of a kernel and $T$ plays the same role as $\frac{1}{h}$.
On $L^2\pa{[0,1]^d}$ we can use bases for which $e_j=f_{j_1}\cdots f_{j_d}$ where $\pa{f_k}_{k=1}^\infty$ is a basis of $L^2\pa{[0,1]}$ and $\pa{j_1,...,j_d}$ plays the role of the index $j$\footnote{Note that there exists a bijection $\N^d\rightarrow \N$.}.
For example, $f_{k}\pa{x} = \sqrt{2}\sin\pa{\pi k x}$ is a basis of $L^2\pa{[0,1]}$.
This gives
\begin{equation*}
  e_{j_1,\ldots,j_d} \pa{x} = 2^{\frac{d}{2}} \prod_{k=1}^d \sin\pa{\pi j_k x_k}.
\end{equation*}
One can check that this defines an orthogonal system \textbf{(Exercise)}.
\\ We define
\begin{equation*}
\begin{split}
W\pa{\beta, L} = &\left\{ \vphantom{\sum_{j_d=1}^\infty}f: \bra{0,1}^d\rightarrow \R \text{ with coefficients } c_{{j_1},\ldots,{j_d}}\text{ w.r.t. } \pa{f_{j_1}, \ldots, f_{j_d}} \right.  \\
&\quad\quad \left. \text{such that } \sum_{j_1=1}^\infty \sum_{j_2=1}^\infty\cdots \sum_{j_d=1}^\infty c_{j_1,...,j_d}^2\pa{j_1^2+...+j_d^2}^\beta\leq L^2 \right\}
\end{split}
\end{equation*}

\begin{remark}
  The $\norm{j}^{2\beta}$ is present due to the fact that we take derivatives of our basis functions defined above till the order of $\beta$.
\end{remark}

{In $L^2\pa{\R^d}$, an analogous condition (with Fourier transform instead of Fourier series) would be:
\begin{equation*}
\int_{\R^d} \abs{\mathcal F[f]\pa{w_1,...,w_d}}^2\pa{\abs{w_1}^2+...+\abs{w_d}^2}^\beta dw \leq L^2.
\end{equation*}}
Note the usual bias-variance decomposition of the mean-squared error,
\begin{equation*}
  \E\bra{\norm{\hat f_X^T - f_X}^2_2} = \underbrace{\norm{f_X^T - f_X }^2_2}_{b^2=\text{Bias}^2} + \underbrace{\E\bra{\norm{\hat{f}_X^T - f_X^T}_2^2}}_{\sigma^2}.
\end{equation*}
Then
\begin{equation*}
\begin{split}
  b^2 &= \sum_{j_1 = T+1}^\infty \cdots \sum_{j_d = T+1}^\infty c^2_{j_1,\ldots, j_d} \\
  &\leq \sum_{j_1 = T+1}^\infty \cdots \sum_{j_d = T+1}^\infty c^2_{j_1,\ldots, j_d} \pa{\pa{\frac{j_1}{T+1}}^2 + \cdots + \pa{\frac{j_1}{T+1}}^2}^\beta \\
  &= \pa{\frac{1}{T+1}}^{2\beta } \sum_{j_1 = T+1}^\infty \cdots \sum_{j_d = T+1}^\infty c^2_{j_1,\ldots, j_d} \pa{j_1^2 + \cdots + j_d^2}^\beta \\
  &\leq \pa{\frac{1}{T+1}}^{2\beta} L^2.
\end{split}
\end{equation*}
Note that $\norm{f_{j_1}\cdots f_{j_d}}_2^2 = \norm{f_{j_1}}_2^2 \cdots \norm{f_{j_d}}_2^2$, which are all $=1$. Then,
\begin{equation*}
\begin{split}
  \sigma^2 &= \E\bra{ \norm{\hat{f}_X^T - f_X^T}_2^2}\\
   &= \E\bra{\sum_{j_1=1}^T \cdots \sum_{j_d=1}^T \pa{\hat{c}_{j_1,\ldots,j_d} - c_{j_1,\ldots, j_d}}^2\norm{f_{j_1}\cdots f_{j_d}}^2_2 }\\
   &= \E\bra{\sum_{j_1=1}^T \cdots \sum_{j_d=1}^T \pa{\hat{c}_{j_1,\ldots,j_d} - c_{j_1,\ldots, j_d}}^2} \\
  &= \sum_{j_1=1}^T \cdots \sum_{j_d=1}^T \Var\pa{\hat c_{j_1,...,j_d}}   \\
  &\leq \sum_{j_1=1}^T \cdots \sum_{j_d=1}^T \frac{2^d}{n}\\
  &\leq \frac{\pa{2 \pa{T+1} }^d }{n}
\end{split}
\end{equation*}
since
\begin{equation*}
\begin{split}
  \Var\pa{\hat c_{j_1,...,j_d}} &= \frac{1}{n^2} \sum_{i=1}^n \Var \pa{f_{j_1} \cdots f_{j_d} \pa{X_i}} \\
  &\leq \frac{1}{n^2} \sum_{i=1}^n \E \bra{f_{j_1}^2 \cdots f_{j_d}^2 \pa{X_i}}\\
  &\leq \frac{2^d}{n},
  \end{split}
\end{equation*}
where we used $f_{j_k}^2 \leq 2$ for our particular basis $f_{j_k}\pa{x} = \sqrt{2}\sin\pa{\pi j_k x}$.
Remember $\sigma^2 \leq \frac{1}{nh^d}$ for kernel estimators.
This gives an upper bound of the order of $n^{-\frac{2\beta}{2\beta +d }}$ if $T = \left\lfloor n^{\frac{1}{2\beta + d}}\right\rfloor$.
\begin{remark}(Nonexaminable content)
    \begin{itemize}
        \item We can work with families of functions which may not be basis functions. We talk about series, dictionaries (machine learning).
        \item In the previous upper bound, the choice of $T$ is infeasible because it depends on $\beta$ which is unknown. 
        \item It it classical to estimate many coefficients $c_j$, for $T$ much larger than before (\emph{e.g.} $\sqrt{n}$) and work with the estimators\begin{equation*}
            \hat f_X^T\pa{x} = \sum_{j_1=1}^T \cdots \sum_{j_d=1}^T \hat \tau(c_{j_1,...,j_d}) e_{j_1}\cdots e_{j_d}\pa{x}.
        \end{equation*}
        where $\tau \propto \frac{\sqrt{\log n}}{n}$. For example 
        \begin{itemize}
            \item $\tau_\rho \pa{x} = \one_{\set{\abs{x}\geq \rho}}$, where $\rho$ is a thresholding function. This is the \textbf{hard} thresholding function.
            \item $\tau_\rho \pa{x} = x \max \pa{1-\frac{\rho}{\abs{x}},0}$. This is the \textbf{soft} thresholding function.
            \end{itemize}
    \end{itemize}

     
\end{remark}
\section{Regression Function Estimation}
\subsection{Introduction: average effect of $X$ on $Y$}
The model for a nonparametric model is $$Y=f(X)+\varepsilon$$, where $\E\bra{\varepsilon|X}=0$ and $\E\bra{\abs{\varepsilon}}<\infty$. The goal is to estimate $f$.
We say it has a random design if $X$ is random, and a fixed design if $X$ is fixed. We will focus on the random design case.
\\First, we define the average effect of $X$ on $Y$ as $\E\bra{f\pa{X}}$ if the expectation is defined.
\\If $f_{y\mid x}$, the conditional density of $Y$ given $X$ exists, it is given by $f_{Y\mid X}\pa{y\mid x} = \frac{f\pa{x,y}}{f_X\pa{x}}$ if $f_X\pa{x}>0$.
Also the conditional expectation function $\E\bra{Y|X=x}$ is given by $$\E\bra{Y|X=x}= \int y f_{Y\mid X}\pa{y\mid x}dy = \frac{\int y f\pa{x,y}dy}{f_X(x)}=\frac{\int y f\pa{x,y}dy}{\int  f\pa{x,y}dy}.$$
A\ natural idea would be to use $$ \hat{f}_X(x) = \frac{1}{nh^2}\sum_{i=1}^n K\pa{\frac{X_i-x}{h}}K\pa{\frac{Y_i-y}{h}},$$ where $K$ is a kernel. 

As an exercise, we can check that $$\int y \hat{f}_{Y,X}\pa{y,x}dy = \frac{1}{nh}\sum_{i=1}^n K\pa{\frac{X_i-x}{h}}Y_i$$ and $$\int \hat{f}_{Y,X}\pa{y,x}dy = \frac{1}{nh}\sum_{i=1}^n K\pa{\frac{X_i-x}{h}}.$$ 
\subsection{Nadaraya-Watson estimator}
This leads to the following estimator called \textbf{Nadaraya-Watson estimator}  $$\hat{f}_X(x) = \frac{\sum_{i=1}^n K\pa{\frac{X_i-x}{h}}Y_i}{\sum_{i=1}^n K\pa{\frac{X_i-x}{h}}}.$$
In practice, dealing with the denominator can be tricky. We propose two ideas to deal with this issue.
\begin{enumerate}
    \item We can work with \textbf{nonnegative} kernels because \begin{equation*}
        \sum Y_i \underbrace{\frac{K\pa{\frac{X_i-x}{h}}}{\sum_{i=1}^n K\pa{\frac{X_i-x}{h}}}}_{\in \bra{0,1}}.
    \end{equation*}
    \item We can use a trimming factor $\rho$ and write $$\hat{f}_X(x) = \frac{\sum_{i=1}^n K\pa{\frac{X_i-x}{h}}Y_i}{\max\pa{\sum_{i=1}^n K\pa{\frac{X_i-x}{h}},\rho}}.$$
\end{enumerate}
Suppose now $\text{supp}\pa{X}=[a,b]$ and $\exists\ m>0\ \text{s.t.}\ f_X(x)\ge m$. Suppose I am interested in $f\pa{b}$ and I use the rectangular kernel \ref{ker:rect}. 
\begin{figure*}[!h]
    \centering
    \includegraphics[width=0.8\textwidth]{figures/kernels.png}
    \caption{Kernels}
\end{figure*}
Then $\hat{f}\pa{b}$ where $\hat{f}$ is the N.W. estimator is biased. But a local polynomial estimator of order $\ge 1$ is consistent and unbiased.
\begin{remark}
    In TD, we will see that we can get a fast rate of convergence with nonnegative kernels (unlike in density estimation).
\end{remark}
\subsection{Local Polynomial Estimation}
\textit{tutorial 3}
\section{Treatment Effects}
\subsection{Setup} We have a dataset $\bra{Y_i,D_i,X_i,Z_i,W_i}_{i=1}^n$.
\begin{itemize}
    \item $D$ is a binary treatment variable, $D\in\bra{0,1}$.
    \item $Y$ is the outcome variable. Here $Y$ is a random variable $Y\in \R$.
    \item $X,Z,W$ are covariates/additional random variables. If we are to make use of them, they need to satisfy the following conditions: if $X=X\pa{0}+D\pa{X\pa{1}-X\pa{0}}$, then $X(0)=X(1)$ a.s.
\end{itemize}
The model equation (for each individual $i$) is $y_i=y\pa{0}(1-D)+y\pa{1}D$. The potential outcome is $y_i\pa{1},y_i\pa{0}$, which are not observed. We can only observe $y_i=y\pa{D_i}$.
\begin{proposition}
    For any $y\in \R$, we have
    \begin{equation*}
        \p\pa{Y(1)\le y \mid D=1} =\E\bra{\1\{Y(1)\le y\}\mid D=1}=\E\bra{\1\{Y\le y\}\mid D=1}
    \end{equation*}
    Similarly, we have $\p\pa{Y(0)\le y \mid D=0} =\E\bra{\1\{Y\le y\mid D=0\}}$.
\end{proposition}
\begin{proof}
    Let $B=\{0\},\{1\}$ or $\{0,1\}$. Then
    \begin{align*}
        \E\bra{\1\{Y\le y\}\1\{D\in B\}} & = \E\bra{\1\{Y(1)D+Y(0)(1-D)\le y\}\1\{D\in B\}}                               \\
                                         & = \E\bra{\1\{Y(1)\le y\}D\1\{D\in B\}}+E\bra{\1\{Y(0)\le y\}(1-D)\1\{D\in B\}}
    \end{align*}
    When $B=\{1\}$, we have
    \begin{align*}
        \E\bra{\1\{Y\le y\}\1\{D=1\}} & = \E\bra{\1\{Y(1)\le y\}D\1\{D=1\}}  +0 \\
                                      & = \E\bra{\1\{Y(1)\le y\}\1\{D=1\}}
    \end{align*}
\end{proof}

On a side note, we have the following lemma:
\begin{equation*}
    \E\bra{Y \mid X\ \text{is some event}}=\frac{\E\bra{Y\1\{X\}}}{\p\{X\}}
\end{equation*}
Therefore, showing that $\E\bra{\1\{Y\le y\}\1\{D=1\}}=\E\bra{\1\{Y(1)\le y\}\1\{D=1\}}$ is equivalent to showing $\E\bra{\1\{Y(1)\le y\}\mid D=1}=\E\bra{\1\{Y\le y\}\mid D=1}$.
\begin{remark}
    To think of it an easier way if that $Y=f(D)$ is a function of $D$ where $f(1)=Y(1)$ and $f(0)=Y(1)$. Naturally, we have $\E\bra{\1\{Y\le y\}\mid D=1}=\E\bra{\1\{f(D=1)\le y\}\mid D=1}=\E\bra{\1\{Y(1)\le y\}\mid D=1}$
\end{remark}

\subsection{Parameters of Interest}
\begin{itemize}
    \item Average treatment effect (ATE): $\tau = \E\bra{Y\pa{1}-Y\pa{0}}$
    \item Conditional average treatment effect (CATE): $\tau(x) =
              \E\bra{Y\pa{1}-Y\pa{0}|X=x}$. It can be useful if we care about the effect of
          the treatment on a specific subgroup of the population.
    \item Average treatment effect on the treated (ATT): $\tau_{\text{ATT}} =
              \E\bra{Y\pa{1}-Y\pa{0}|D=1}$
    \item Average treatment effect on the untreated (ATU): $\tau_{\text{ATU}} =
              \E\bra{Y\pa{1}-Y\pa{0}|D=0}$
    \item Conditional average treatment effect on the treated (CATT):
          $\tau_{\text{ATT}}(x) = \E\bra{Y\pa{1}-Y\pa{0}|D=1,X=x}$
\end{itemize}

Depending on the design of the experiment we specify the identification
requirements for the parameters of interest.

\subsection{Randomized experiment}
The treatment is randomly assigned to individuals, which is equivalent to
saying that \begin{equation}\label{eq:random_experiment}
    Y(0),Y(1),W \indep D
\end{equation}
This is a rather strong assumption. Under this assumption, $\p_{Y(1),X}$ and $\p_{Y(0),X}$ are identified.

We introduce an estimator for the ATE under the random experiment assumption.
\paragraph{Estimator (no stratum)} The group mean difference estimator is defined
as
\begin{equation*}
    \widehat{ATT}=\frac{1}{n_T}\sum_{i\in T} Y_i-\frac{1}{n_C}\sum_{i\in C} Y_i
\end{equation*}
The variance is estimated as
\begin{equation*}
    \widehat{\text{Var}\pa{\widehat{ATT}}}=\frac{1}{n_T}\hat{\sigma}_T^2+\frac{1}{n_C}\hat{\sigma}_C^2
\end{equation*}
Asymptotic normality holds under ususal assumptions.
The group mean difference estimator is the \textbf{least square estimator} of $\alpha_1$ in $Y=\alpha_0+\alpha_1D+\epsilon$ with $\E[\epsilon]=\E[\epsilon\mid D]=0$.
We can not test the assumption \ref{eq:random_experiment} directly. However, we can test its implication (necessary condition, so to speak). For example, it implies that $\E[D\mid W]=0$ (???). Therefore, we can test this implication by regressing $W$ (or any function of $W$) on $D$ with $W=\alpha_0+\alpha_1 D+\epsilon$ ( $\E[\epsilon]=\E[\epsilon\mid D]=0$) and test $H_0:\alpha_1=0$.
\paragraph{Estimator (with stratum)} The group mean difference estimator is defined
as
\begin{equation*}
    \widehat{ATT}=\sum_{s=1}^S \widehat{ATT}_s\frac{n_s}{n}
\end{equation*}
The variance is estimated as
\begin{equation*}
    \widehat{\text{Var}\pa{\widehat{ATT}}}=\sum_{s=1}^S \frac{n_s}{n}\hat{\sigma}_s^2
\end{equation*}

\begin{remark}
    The more the strata we have, the more precise the estimator is (the variance is smaller?). However, we want to have observations in both $T$ and $C$ in each stratum, so they can not be too small.
\end{remark}

\subsection{Randomized experiment based on observables}
Now we introduce a new notation for the purpose of another assumption.
\begin{definition}[Propensity score]
    The propensity score is defined as the conditional probability of receiving the treatment given the covariates, that is \[\pi(x)=\p(D=1\mid X=x)\]
\end{definition}
\begin{remark}
    Later we will build estimators using propensity score, called \textbf{inverse propensity score weighting (IPSW) estimator}.
\end{remark}
\begin{proposition}
    For all $x\in supp(X)$ and $\pi \in supp(\pi(X))$, we have \begin{equation*}
        \E\bra{\1\set{X\le x} \mid D=1, \pi(X)=\pi}= \E\bra{\1\set{X\le x} \mid D=1, X=x}
    \end{equation*} Note that we define $\p(X\le x \mid D=1, \pi(X)=\pi)$ as $\E\bra{\1\set{X\le x} \mid D=1, \pi(X)=\pi}$.
\end{proposition}

Sometimes, we can not completely randomize the experiment, but we can randomize
the treatment conditional on some observables, which is equivalent to saying
that
\begin{equation}\label{eq:random_experiment_observables}
    Y(0),Y(1),W \indep D \mid X
\end{equation} This is called \textbf{unconfoundedness}
A weaker version is conditional mean independence:
$\E\bra{Y(1)\mid D,X}=\E\bra{Y(1)\mid X}$ and $\E\bra{Y(0)\mid D,X}=\E\bra{Y(0)\mid X}$.
Unconfoundedness implies conditional mean independence.

\paragraph{Common support} The propensity score $\pi(x)$ continuous and bounded between 0 and 1 for all
$x\in \text{supp}(X)$. We add the assumption of common support in order to
develop the following equality, and to build new formula for estimator based on
these equalities.

For the following equalities, we need the weaker assumption -- conditional mean
independence $\E\bra{Y(1)\mid D,X}=\E\bra{Y(1)\mid X}$
\begin{proposition}[Conditional treatment effect]
    For all $x\in supp(X)$ and $\pi \in supp(\pi(X))$, we have
    \begin{equation}
        \begin{split}
            ATE(x)=ATT(x)= ATU(x) & = \E\bra{Y(1)-Y(0) \mid X=x}                          \\
            & = \E\bra{Y(1)\mid X=x, D=1}-\E\bra{Y(0)\mid X=x, D=0} \\
            & = \E\bra{Y\mid X=x, D=1}-\E\bra{Y\mid X=x, D=0}       \\
            & = \mu_1(x)-\mu_0(x)
        \end{split}
    \end{equation}
    where
    \begin{align}
        \E\bra{Y\mid X=x, D=1} & =\frac{\E[DY|X=x]}{\pi(x)}       \\
        \E\bra{Y\mid X=x, D=0} & =\frac{\E[(1-D)Y|X=x]}{1-\pi(x)}
    \end{align}

\end{proposition}

From the above equalities, we give another proposition
\begin{proposition}[Unconditional treatment effect]
    By definition, we have
    \begin{align}
        ATE & =\E\bra{ATE(X)}=\E\bra{\frac{DY}{\pi(X)}-\frac{(1-D)Y}{1-\pi(X)}}            \\
        ATT & =\E\bra{ATT(X)\mid D=1}=\frac{1}{P(D=1)}\E\bra{\frac{(D-\pi(X))Y}{1-\pi(X)}} \\
    \end{align}
\end{proposition}

\paragraph{Estimator} An estimator for ATT and ATE is defined as
\begin{align}
    \widehat{ATE} & = \frac{1}{n}\sum_{i=1}^n \pa{\hat{\mu}_1(X_i)-\hat{\mu}_0(X_i)}              \\
    \widehat{ATT} & =\frac{\sum_{i=1}^n D_i(\hat{\mu}_i(X_i)-\hat{\mu}_0(X_i))}{\sum_{i=1}^n D_i}
                  &
\end{align}
The Inverse propensity score weighting (IPSW) estimator for ATT and ATE is defined as
\begin{align}
    \widehat{ATE} & = \frac{1}{n}\sum_{i=1}^n \pa{\frac{D_iY_i}{\hat{\pi}(X_i)}-\frac{(1-D_i)Y_i}{1-\hat{\pi}(X_i)}} \\
    \widehat{ATT} & =\frac{1}{\sum_i D_i}\sum_{i=1}^n \frac{(D_i-\hat{\pi}(X_i))Y_i}{1-\hat{\pi}(X_i)}
\end{align}

In order to analyze the Asymptotic Properties of the IPSW estimator, it is
convenient to impose the following assumptions on the propensity score
$\pi(x)$: There exists a $\eta\in (0,1)$ such that for all $x\in supp(X)$, we
have $\eta\le \pi(x)\le 1-\eta$. This is called the \textbf{no overlap}
assumption.

The estimators can sometimes be asymptotically normal if the functions can be
well estimated. This is relatively manageable if the rate of convergence is
parametric (for example we know the functional form up to a finite dimensional
parameter). When the functions are estimated nonparametrically this is more
difficult because of possible slow rate of estimation of those.

    {\color{red} I am completely oblivious to any assumption related to support, the nuances between surely and almost surely.}

\subsection{Regression discontinuity design}
\subsubsection{Sharp RDD}
\paragraph{Preliminaries}
Previously, we consider $D$ as a binary variable and impose certain conditions on whether $D_i=1$ or $D_i=0$ (for each individual). Now we specify how $D$ is determined by a continuous variable $X$, that is  $$D_i=\1\set{X_i\ge c}$$
where $c$ is a known threshold. The idea is that the treatment is assigned based on the value of $X$. We can think of $X$ as a score, and $D$ is assigned to those who score above a certain threshold.

For example, in the context of education, $X$ can be the score of a student in a standardized test, and $D$ is whether the student is admitted to a college. The threshold $c$ is the cutoff score for admission.

\paragraph{Conditions}
Recall the definition of \emph{unconfoundedness}:
\begin{definition}[unconfoundedness]
    The treatment $D$ is unconfounded with the potential outcome $Y$ given $X$ if 
    \begin{equation*}
        Y\pa{1},Y\pa{0}\indep D\mid X
    \end{equation*}
\end{definition}

It is easy to see that since $D$ is determined by $X$, the potential outcome $Y\pa{1},Y\pa{0}$ are independent of $D$ given $X$. (Given $X$, $D$ is already determined, thus a constant.) Therefore, the unconfoundedness assumption is satisfied.


Since $X$ is a continuous variable, we make an assumption on the \textbf{average potential outcome} $\E\bra{Y(j)\mid X=x}$ for $j=0,1$. We assume that the average potential outcome is continuous at the threshold $c$. For example, for those who have a test score slightly above and below the threshold, the average potential earning $Y(1),Y(0)$ is similar.
\begin{remark}
    We assume that $\E\bra{Y(j)\mid X=x}$ is continuous at the threshold $c$ but not $\E\bra{Y\mid X=x}$ at $c$.
\end{remark}

Let us now check conditional ATE at the point $c$. Previously, we define the conditional ATE as \begin{equation*}
    \begin{split}
        \tau(x) &= \E\bra{Y\pa{1}-Y\pa{0}\mid X=x} \\&= \E\bra{Y\pa{1}\mid X=x}-\E\bra{Y\pa{0}\mid X=x} \\& =\underbrace{\E\bra{Y\mid D=1,X=x}-\E\bra{Y\mid D=0,X=x}}_{\text{by unconfoundedness thus mean independence}} \\
    \end{split}
\end{equation*}
Therefore, \begin{equation}
    \begin{split}
        \tau(c) &= \lim_{x\to c^+}\E\bra{Y\mid D=1,X=x}-\lim_{x\to c^-}\E\bra{Y\mid D=0,X=x}\\ &= \lim_{x\to c^+} \E\bra{Y\mid X=x}-\lim_{x\to c^-}\E\bra{Y\mid X=x} 
    \end{split}
\end{equation}

\paragraph{Estimation}
To estimate $\E\bra{Y\mid X=x}$, we can use local polynomial of order 0 (Nadaraya-Watson estimator) or more:
\begin{align*}
    \pa{\hat{\alpha}_1,\hat{\beta}_1} = \argmin_{\alpha,\beta}\sum_{i=1,X_i\ge c}^n K\pa{\frac{X_i-x}{h}}\pa{Y_i-\alpha-\beta\pa{X_i-c}}^2 \\
    \pa{\hat{\alpha}_0,\hat{\beta}_0} = \argmin_{\alpha,\beta}\sum_{i=1,X_i< c}^n K\pa{\frac{X_i-x}{h}}\pa{Y_i-\alpha-\beta\pa{X_i-c}}^2 
\end{align*}
Then we estimate $\tau(c)$ by $\hat{\tau}(c) = \hat{\alpha}_1-\hat{\alpha}_0$.
\subsubsection{Fuzzy RDD}
\paragraph{Preliminaries}
First, let's recall the definition of conditional mean independence:
\begin{definition}[Conditional mean independence]
    We say $U$ and $V$ are conditionally mean independent given $X$ if \begin{equation*}
        \E\bra{\phi(U)\psi(V)\mid X} = \E\bra{\phi(U)\mid X}\E\bra{\psi(V)\mid X}
    \end{equation*}
\end{definition} Naturally local conditional mean independence in a neighborhood $\caln$ is defined as \begin{equation*}
    \E\bra{\phi(U)\psi(V)\mid X=x} = \E\bra{\phi(U)\mid X=x}\E\bra{\psi(V)\mid X=x}
\end{equation*} for almost every $x\in \caln$. 
\paragraph{Condition}
Now we are ready to move from \emph{sharp RDD} to \emph{fuzzy RDD}. In the fuzzy RDD, the treatment $D$ is not exactly determined by $X$ but instead satisfies the following condition to create a discontinuity at $c$:
The propensity score function $\pi(x)$ is continuous on $(c-\epsilon,c)$ and $(c,c+\epsilon)$ for some $\epsilon>0$ and $\lim_{x\to c^+}\pi(x)\neq \lim_{x\to c^-}\pi(x)$.
We also loosen the mean independence condition to local conditional mean independence in a neighborhood $\caln$ of $c$.
\begin{remark}
    The fuzzy RDD is more general than the sharp RDD. 
\end{remark}
Since we depart from sharp RDD, the conditional ATE at $c$ is now defined as the following:
\begin{proposition}\begin{equation*}
    \tau(c) = \frac{\lim_{x\to c^+}\E\bra{Y\mid D=1,X=x}-\lim_{x\to c^-}\E\bra{Y\mid D=0,X=x}}{\lim_{x\to c^+}\pi(x)-\lim_{x\to c^-}\pi(x)}
\end{equation*}
\end{proposition}
\begin{proof}
    
\end{proof}
\paragraph{Estimation}
We can make use of local polynomial estimator to estimate the denominator and the numerator separately for $tau(c)$.


\subsection{Instrumental variable}

In this section, we are in the case of selection on \emph{unobservables}. We have binary treatment $D$, covariates $X$, and a binary instrument $Z$, such that treatment is assigned based on $Z$ (and maybe $X$). But the assigned treatment may not be taken by the individual (imperfect compliance). We have the following model:
\begin{equation}
    \begin{split}
        Y &= Y(0,0)+Z\pa{Y(1,0)-Y(0,0)}+D\pa{Y(0,1)-Y(0,0)} \\
        &+ DZ\pa{Y(1,1)-Y(0,1)-Y(1,0)+Y(0,0)}
    \end{split}
\end{equation}
and \begin{equation}
    D=D(0)+Z\pa{D(1)-D(0)}
\end{equation}
\paragraph{Preliminaries}
We define the following:
\begin{itemize}
    \item One-sided compliance: $P\pa{D(0)=0}=0$ which means there is no always taker or defiers. 
    \item two-sided compliance: $P\pa{D(0)=0}\in (0,1)$ and $P\pa{D(1)=1}\in (0,1)$.
\end{itemize}
\paragraph{Condition}
We need to impose that the assignment $Z$ is independent of the potential outcome $Y(z,d)$ and the treatment $D$ (given $X$). From now on we omitted the conditioning on $X$ for simplicity. This is the \emph{exclusion restriction} assumption. It is similar to $\E\bra{\epsilon\mid X,Z}=0$ assumption in standard linear IV model.

\begin{definition}[Local average treatment effect (LATE)]
    The local average treatment effect is defined as \begin{equation*}
        \tau_{\text{LATE}} = \E\bra{Y(Z,1)-Y(Z,0)\mid D(1)-D(0)=1}
    \end{equation*}
    which is the average treatment effect for the compliers.
\end{definition}


\subsubsection{One-sided compliance}
Instead of discussing ATE, we define a new set of parameters called \emph{Intention to treat} (ITT).
\begin{definition}[Intention to treat (ITT)]
    The intention to treat on treatment is defined as \begin{equation*}
        \text{ITT}_D = \E\bra{D(1)-D(0)\mid Z=1}
    \end{equation*}
    The intention to treat on outcome is defined as \begin{equation*}
        \text{ITT}_Y = \E\bra{Y(1,D(1))\mid Z=1}-\E\bra{Y(0,D(0))\mid Z=0}
    \end{equation*}
    The intention to treat on outcome for compilers is defined as \begin{equation*}
        \text{ITT}_{Y\mid D(1)-D(0)=1} = \E\bra{Y(1,1)-Y(0,1)\mid D(1)-D(0)=1}
    \end{equation*}
\end{definition} 
\paragraph{Condition}
Under the one-sided compliance, we need to make the following assumption:
\begin{itemize}
    \item If $D(1)=0$ (never taker), then $Y(1,0)=Y(0,0)$.
    \item If $D(1)=1$ (compiler), then $Y(0,d)=Y(1,d)$ for $d=0,1$.
\end{itemize} 
Because every individual is either a never taker or a compiler, we have the following:$$Y(z,d)=Y(d)$$ This is sometimes called the an \textit{exclusion restriction} assumption.

Then it can be shown that \begin{equation*}
    \text{ITT}_Y = \text{ITT}_{Y,CO}\text{ITT}_D
\end{equation*}

Recall that LATE is the average treatment effect for the compilers. And under the condition mentioned above, 
\begin{align*}
    \tau_{\text{LATE}} &= \E\bra{Y(Z,1)-Y(Z,0)\mid D(1)-D(0)=1}\\
    \text{ITT}_{Y\mid {D(1)-D(0)=1}}&=\E\bra{Y(1,1)-Y(0,0)\mid D(1)-D(0)=1}
\end{align*}
That is, \begin{equation*}
    \tau_{\text{LATE}} = \text{ITT}_{Y\mid {D(1)-D(0)=1}}
\end{equation*}
Therefore,
\begin{equation*}
\tau_{\text{LATE}} = \frac{\text{ITT}_Y}{\text{ITT}_D} =  \frac{\E\bra{Y \mid Z=1}-\E\bra{Y\mid Z=0}}{\E\bra{D\mid Z=1}-\E\bra{D\mid Z=0}}
\end{equation*}

\subsubsection{Two-sided compliance}
\paragraph{Condition} This is a natural assumption. 
\begin{itemize} 
    \item For never takers, $Y(0,0)=Y(1,0)$.
    \item For always takers, $Y(0,1)=Y(1,1)$.
    \item For compliers (and defiers), $Y(1,d)=Y(0,d)$.
\end{itemize}
We also need to impose the \emph{monotonicity} assumption. It states that the instrument $Z$ has a monotonic effect on the treatment $D$. That is, $D(1)\ge D(0)$ a.s. or $D(1)\le D(0)$ a.s.

Under these conditions, we also have \begin{equation*}
    \tau_{\text{LATE}} = \frac{\E\bra{Y \mid Z=1}-\E\bra{Y\mid Z=0}}{\E\bra{D\mid Z=1}-\E\bra{D\mid Z=0}}
\end{equation*}
It can be shown that if there is no defiers, then $\E\bra{D\mid Z=1}-\E\bra{D\mid Z=0} = \p\pa{D(1)-D(0)=1}$.

\paragraph{Estimation}
We can estimate $\tau_{\text{LATE}}$ by \begin{equation*}
    \frac{\E\bra{Y \mid Z=1}-\E\bra{Y\mid Z=0}}{\E\bra{D\mid Z=1}-\E\bra{D\mid Z=0}}=\frac{Cov(Y,Z)}{Cov(D,Z)}
\end{equation*}

\subsection{Estimation methods: (Augmented) Inverse probability Weighting (AIPW)}
    

\newpage


\end{document}
