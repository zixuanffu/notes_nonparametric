\paragraph{Parametric VS Non-Parametric Models}
If $\zeta=(\gamma,p_\theta)$ where $\gamma \in \R^{d_\gamma}, \theta \in
    \R^{d_\theta}$, then the model is parametric. If $\zeta=(\gamma,p)$ where
$\gamma \in \R^{d_\gamma}, p$ is a distribution, then the model is
non-parametric. When we care about $\psi^*=\Psi(\zeta^*)$, we are in the
framework of semi-parametric models.
\begin{example}
    The model is $Y=\alpha+\beta X+\varepsilon$ where $\zeta=(\alpha,\beta,p_{\varepsilon,X})$ and $\psi^*=(\alpha^*,\beta^*)$. We are in semi-parametric models because $\alpha,\beta$ are finite-dimensional and $p_{\varepsilon,X}$ is infinite-dimensional.
\end{example}

