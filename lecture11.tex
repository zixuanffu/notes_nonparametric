\section{Treatment Effects}
\subsection{Setup} We have a dataset $\bra{Y_i,D_i,X_i,Z_i,W_i}_{i=1}^n$ following i.i.d. from a
joint distribution.
\begin{itemize}
    \item $D$ is a binary treatment variable, $D\in\bra{0,1}$.
    \item $Y$ is the outcome variable. Here $Y$ is a random variable $Y\in \R$.
    \item $X,Z,W$ are covariates/additional random variables.
\end{itemize}
The model equation (for each individual $i$) is $y_i=y\pa{0}(1-D)+y\pa{1}D$. The potential outcome is $y_i\pa{1},y_i\pa{0}$, which are not observed. We can only observe $y_i=y\pa{D_i}$.

\subsection{Parameters of Interest}
\begin{itemize}
    \item Average treatment effect (ATE): $\tau = \E\bra{Y\pa{1}-Y\pa{0}}$
    \item Conditional average treatment effect (CATE): $\tau(x) =
              \E\bra{Y\pa{1}-Y\pa{0}|X=x}$. It can be useful if we care about the effect of
          the treatment on a specific subgroup of the population.
    \item Average treatment effect on the treated (ATT): $\tau_{\text{ATT}} =
              \E\bra{Y\pa{1}-Y\pa{0}|D=1}$
    \item Average treatment effect on the untreated (ATU): $\tau_{\text{ATU}} =
              \E\bra{Y\pa{1}-Y\pa{0}|D=0}$
    \item Conditional average treatment effect on the treated (CATT):
          $\tau_{\text{ATT}}(x) = \E\bra{Y\pa{1}-Y\pa{0}|D=1,X=x}$
\end{itemize}

\subsection{Identification}
We need to impose some assumptions in order to identify the parameters.
\begin{assumption}
    $\p\pa{D=1}\in \pa{0,1}$
\end{assumption}
\begin{assumption}
    The covariates $X,Z,W$ are such that if $X=X(0)+D\pa{X(1)-X(0)}$, then $X(1)=X(0)$.
\end{assumption}

\begin{assumption}
    The potential outcome $Y\pa{1},Y\pa{0}$ are independent of $D$ given
\end{assumption}
\begin{assumption}
    The potential outcome $Y\pa{1},Y\pa{0}$ are independent of $D$ given $X,Z,W$
\end{assumption}
We introduce a new notation for the purpose of another assumption.
\begin{definition}[Propensity score]
    The propensity score is defined as the conditional probability of receiving the treatment given the covariates, that is \[\pi(x)=\p(D=1\mid X=x)\]
\end{definition}
\begin{remark}
    Later we will build estimators using propensity score, called \textbf{inverse propensity score weighting (IPSW) estimator}.
\end{remark}
\begin{assumption}[Common support]\label{ass:comm_supp}
    The propensity score $\pi(x)$ continuous and bounded between 0 and 1 for all $x\in \text{supp}(X)$
\end{assumption}
\begin{assumption}[Mean independence]\label{ass:mean_indep}
    The potential outcome $Y\pa{1},Y\pa{0}$ are independent of $D$ given $X,Z,W$, that is \[\E\bra{Y(d)\mid D,X}=\E\bra{Y(d)\mid X}\]
\end{assumption}
Some proposition
\begin{proposition}
    For any $Y\in \R$, we have that
    \begin{enumerate}
        \item the expectation of $\1_{\set{Y(1)<y}}$ conditional on $D=1$ equals to the
              expectation of $\1_{\set{Y<y}}$ conditional on $D=1$, that is
              \begin{equation*}
                  \E\bra{\1_{\set{Y(1)<y}}\mid D=1} = \E\bra{\1_{\set{Y<y}}\mid D=1}
              \end{equation*}
        \item By mean independence, $\tau(x) =
                  \E\bra{Y\pa{1}-Y\pa{0}|X=x}=\E\bra{Y\pa{1}|X=x,D=1}-\E\bra{Y\pa{0}|X=x,D=0}$
        \item $\E\bra{Y\pa{1}|X=x,D=1}=$
    \end{enumerate}
\end{proposition}
\begin{proof}
    Because \[
        \1\{Y<y\}=\1\{Y(0)\le y\}(1-D) +\1\{Y(1)<Y\}D \]
    We have that
    \begin{align*}
    \end{align*}
\end{proof}

\begin{proposition}
    Under the assumptions \ref{ass:comm_supp} and \ref{ass:mean_indep}, it can be shown that the conditional ATE(x) equals to ATT(x) and ATU(X), that is \[\tau(x)=\tau_{\text{ATT}}(x)=\tau_{\text{ATU}}(x)\]
\end{proposition}
\begin{proof}

\end{proof}